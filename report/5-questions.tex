%---------------------------------------------------------------
\section{Research Questions}\label{research-questions}

%Note: the VAST challenge is focused on visual analytics and graphical figures should be included with your response to each question. Please include a reasonable number of figures for each question (no more than about 6) and keep written responses as brief as possible (around 250 words per question). Participants are encouraged to new visual representations rather than relying on traditional or existing approaches.

% Question 1
\vspace{1em}
\noindent\textbf{Question 1:} \textit{FishEye analysts have long wanted to better understand the flow of commercially caught fish through Oceanus’s many ports. But as they were loading data into CatchNet, they discovered they had purchased the wrong port records. They wanted to get the ship off-load records, but they instead got the port-exit records (essentially trucks/trains leaving the port area). Port exit records do not include which vessel that delivered the products. Given this limitation, develop a visualization system to associate vessels with their probable cargos. Which vessels deliver which products and when? What are the seasonal trends and anomalies in the port exit records?}

\subsubsection*{Answer}
% Write your answer here (approx 250 words)

To resolve the ambiguity caused by the absence of direct off-load records, we implemented the analytic data association method described in Section~\ref{sec:cargo_attribution}. This technique creates a ``suspected\_vessels'' attribute by correlating the timing and tonnage of port-exit records with vessel arrivals, forming the backbone of our Harbor Inspector view.

The central component of this view is the Mirror Plot, designed to simultaneously visualize seasonal trends and isolate probable vessel-cargo connections. The upper axis aggregates daily cargo exports, revealing a distinct seasonal pulse where volumes surge in late summer and peak in October and November. Notably, anomalies in regulated species exports frequently align with these trade peaks, suggesting that illicit actors strategically mask their activities within high-volume periods – a pattern particularly evident in Lomark and South Paackland. Additionally, the visualization captures regional specializations, such as Haacklee's distinct dominance in the Tuna market.

The lower axis addresses the attribution challenge by plotting vessel arrivals aligned with these export events. By integrating a color-coded alert system, the visualization distinguishes standard traffic from high-risk vessels (highlighted in red). This visual linkage bridges the granular data gap, allowing analysts to trace a specific cargo batch back to a limited set of suspect vessels, which can then be forensically verified in the Trajectory Analyzer view (Section~\ref{sec:compare_trajectories}).

Figure~\ref{fig:south_paackland_activity} illustrates the Mirror Plot for South Paackland Harbor, showcasing both the seasonal export surges and the corresponding vessel associations.

\begin{figure}[h]
    \centering 
    \includegraphics[width=0.9\linewidth]{img/south_paackland_mirror_plot.png} 
    \caption{Mirror Plot for South Paackland Harbor showing seasonal export trends and vessel-cargo associations.} 
    \label{fig:south_paackland_activity} 
\end{figure}

% Question 2
\vspace{1em}
\noindent\textbf{Question 2:} \textit{Develop visualizations that illustrate the inappropriate behavior of SouthSeafood Express Corp vessels. How do their movement and catch contents compare to other fishing vessels? When and where did SouthSeafood Express Corp vessels perform their illegal fishing? How many different types of suspicious behaviors are observed? Use visual evidence to justify your conclusions.}

\subsubsection*{Answer}
% Write your answer here (approx 250 words)
To analyze the behavior of SouthSeafood Express Corp vessels, we utilized the Traffic Explorer for broad spatial verification (Section~\ref{sec:traffic_explorer}) and the Trajectory Analyzer view for detailed temporal analysis (Section~\ref{sec:compare_trajectories}).

While the automated cargo attribution did not flag SouthSeafood vessels as prominently as other illicit actors, a manual inspection of their trajectories reveals distinct suspicious behaviors when compared to compliant vessels. We premise our analysis on the baseline that compliant fishing vessels generally maintain direct routes, avoiding restricted areas unless strictly necessary for transit.

\emph{Snapper Snatcher} clearly breaks this compliant pattern. As illustrated in Figure~\ref{fig:snapper_comparison}, unlike the control vessel (\textit{Whiting Wrangler}), Snapper Snatcher records confirmed presence within the Ghoti Preserve (indicated by the distinct green markers in February and March). Furthermore, the vessel displays erratic routing, abruptly switching fishing grounds from the Cod Table to Wrasse Beds and making unusually extended stops at the ``Exit East'' buoy. These deviations are inconsistent with standard fishing operations and suggest potential poaching or transshipment activities.

\begin{figure}[h] 
    \centering 
    \includegraphics[width=0.9\linewidth]{img/snapper_whiting.png} 
    \caption{Comparison of Snapper Snatcher (SouthSeafood Express Corp) showing incursions into Ghoti Preserve vs. the compliant Whiting Wrangler.} \label{fig:snapper_comparison} 
\end{figure}

\emph{Roach Robber} presents a more subtle case (Figure~\ref{fig:roach_comparison}). While its route appears more predictable than the one of Snapper Snatcher, it exhibits suspicious loitering at Nav 1, a buoy situated directly over the Gothi Preserve. A compliant vessel (e.g., \textit{Tiger Muskellunge Master}) typically bypasses such markers or adheres to standard navigation channels (e.g., Nav C). This proximity to the preserve boundary, combined with the lack of clear fishing activity elsewhere during those windows, could indicate ``edge-fishing'' or coordination with others vessels operating inside the restricted zone.

\begin{figure}[h] 
    \centering 
    \includegraphics[width=0.9\linewidth]{img/roach_tiger.png} 
    \caption{Comparison of Roach Robber (SouthSeafood Express Corp) showing suspicious loitering at Nav 1 vs. a compliant vessel.} 
    \label{fig:roach_comparison} 
\end{figure}

Although the lack of granular catch data prevents a direct comparison of the cargos (as neither Snapper Snatcher nor Roach Robber was attributed an illegal cargo), the spatial anomalies (specifically the preserve incursions and buoy loitering) provide sufficient visual evidence to warrant the deeper investigation detailed in Question 3.

% Question 3
\vspace{1em}
\noindent\textbf{Question 3:} \textit{To support further Fisheye investigations, develop visual analytics workflows that allow you to discover other vessels engaging in behaviors similar to SouthSeafood Express Corp’s illegal activities? Provide visual evidence of the similarities.}

\subsubsection*{Answer}
% Write your answer here (approx 250 words)
To discover other vessels mimicking SouthSeafood Express Corp’s illicit strategies, we developed a behavioral profiling workflow based on the "signatures" identified in Question 2: (1) direct incursions into restricted zones (Ghoti Preserve) and (2) non-transit loitering at specific buoys (Exit East).

As a first step, we employed the Traffic Explorer to filter the fleet for vessels exhibiting spatial overlaps with the known SouthSeafood offenders. By pinning the \emph{Snapper Snatcher} as a reference (Figure~\ref{fig:traffic_filtering}), we visually scanned for vessels that frequented the same sectors during identical timeframes. This broad filtering process highlighted the \emph{Catfish Capturer} as a high-priority suspect due to its overlapping presence in both the Ghoti Preserve and Exit East buoy areas.

\begin{figure}[h] 
    \centering 
    \includegraphics[width=0.9\linewidth]{img/snapper_catfish_traffic.png} 
    \caption{Traffic Explorer filtering process. Highlighting Catfish Capturer (yellow) based on spatial overlaps with the pinned Snapper Snatcher.} 
    \label{fig:traffic_filtering} 
\end{figure}

We then validated this suspect using the Trajectory Analyzer view (Figure~\ref{fig:similarity_verification}). The visual alignment is striking. Prior to the interdiction (left of the red dotted line), Catfish Capturer mirrors Snapper Snatcher with high precision. Both vessels record presence in the Ghoti Preserve (green markers) during late February and May, followed by identical loitering patterns at the Exit East buoy (brown markers), suggesting a coordinated protocol or shared illegal route.

Finally, we examined post-interdiction behavior to assess the impact of enforcement actions. With the banning of SouthSeafood Express Corp in May, Catfish Capturer drastically altered its behavior. As seen in the right half of Figure~\ref{fig:similarity_verification}, the vessel immediately ceased all Ghoti Preserve and Exit East activity. Instead, it shifted entirely to legitimate operations, recording dense fishing activity in the Wrasse Beds (indicated by the continuous blue blocks at the bottom of the chart). This may suggest that the Catfish Capturer vessel was successfully deterred and returned to compliant fishing.

\begin{figure}[h] 
    \centering 
    \includegraphics[width=0.9\linewidth]{img/snapper_catfish_trajectories.png} 
    \caption{Visual Evidence of Similarity and Reform. Left: Snapper Snatcher (illegal reference). Right: Catfish Capturer mirrors the illegal Ghoti/Exit East stops (green/brown markers) before the crackdown, but shifts to legal Wrasse Beds fishing (blue bars) immediately after.} 
    \label{fig:similarity_verification} 
\end{figure}

% Question 4
\vspace{1em}
\noindent\textbf{Question 4:} \textit{How did fishing activity change after SouthSeafood Express Corp was caught? What new behaviors in the Oceanus commercial fishing community are most suspicious and why?}

\subsubsection*{Answer}
% Write your answer here (approx 250 words)

The Trajectory Analyzer view enables a temporal analysis of fishing activity before and after the interdiction of SouthSeafood Express Corp (marked by the vertical red dotted line). Post-interdiction, we observe a notable shift in fishing patterns across the Oceanus fleet. Several vessels exhibited increased caution, avoiding previously frequented areas such as the Ghoti Preserve. However, traffic in the broader region did not diminish; instead, it redistributed, suggesting that illegal activities persisted and adapted to the new enforcement landscape. Two primary suspicious behaviors emerged: displacement to less-monitored protected areas and the adoption of ``edge-fishing'' tactics.

As shown in Figure~\ref{fig:displacement}, vessels that previously ignored the Ghoti Preserve suddenly began targeting the Nemo Reef Ecological Preserve immediately after the crackdown. The \emph{White Marlin Master} (Figure~\ref{fig:displacement}, left) and the \emph{Marlin Master} (Figure~\ref{fig:edge_fishing}, right) both show a distinct lack of preserve activity prior to the event, followed by repeated incursions into Nemo Reef afterward. This suggests a coordinated shift to exploit a less-monitored protected area.

\begin{figure}[h] 
    \centering 
    \includegraphics[width=0.9\linewidth]{img/whitemarlin_bluemarlin.png} 
    \caption{Behavioral divergence post-interdiction. Left: White Marlin Master shifts operations into Nemo Reef. Right: Blue Marlin Bandit ceases all port calls to loiter at Nav C.} 
    \label{fig:displacement} 
\end{figure}

Other vessels adopted tactics to avoid inspection. The \emph{Blue Marlin Bandit} (Figure~\ref{fig:displacement}, right) completely ceased docking at ports (Himark/Haacklee) after the interdiction. Instead, it shifted to the Wrasse Beds and began loitering at Nav C, likely awaiting transshipment vessels to offload catch without entering a harbor.

Similarly, the \emph{Bluefin Tuna Bandit} (Figure~\ref{fig:edge_fishing}, left) adopted "edge fishing" tactics. After the crackdown, it began frequenting Nav 1, a buoy positioned directly on the boundary of the Ghoti Preserve. This suggests the vessel is skimming the edge of the restricted zone, dipping in only briefly or luring fish out, essentially testing the limits of the new enforcement protocols.

\begin{figure}[h] 
    \centering \includegraphics[width=0.9\linewidth]{img/bluefin_marlinmaster.png} \caption{Evasive tactics. Left: Bluefin Tuna Bandit begins ``edge fishing'' at Nav 1 (boundary of Ghoti Preserve). Right: Marlin Master confirms the fleet-wide shift to poaching in Nemo Reef.} 
    \label{fig:edge_fishing} 
\end{figure}