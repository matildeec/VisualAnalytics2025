%---------------------------------------------------------------
\section{Conclusion and Reflections}

This report has detailed the design, implementation, and application of a visual analytics system aimed at investigating illegal fishing activities within the fictional nation of Oceanus. Through a combination of novel visualization techniques and a user-centered design approach, we have developed a platform that effectively addresses the complex questions posed by the VAST Challenge 2024.

Reflecting on the development process, the primary challenge lay in the ambiguity and scale of the provided datasets. The lack of direct vessel-to-cargo linkages required not just data visualization, but active data reconstruction, which we addressed through inferred associations as described in Section~\ref{sec:cargo_attribution}. Furthermore, the sheer volume of vessel traffic and port transactions necessitated a design that could distill vast amounts of information without overwhelming the analyst. Our solution, splitting the interface into distinct conceptual views (Traffic, Harbor, Trajectory), proved essential. This compartmentalization established a workflow that moves from broad surveillance to granular forensic analysis, managing the data's complexity effectively.

Looking forward, the platform could be enhanced through tighter integration of these views. Currently, identifying the ``geographical justification'' for a vessel's stop requires the user to mentally bridge the gap between the static Traffic Explorer map and the linear Compare Trajectories timeline. Future iterations should implement a unified geospatial-temporal interface to visualize the proximity, periodicity, and spatial context of stops directly within the trajectory view, reducing the cognitive load of cross-referencing.

Additionally, the system's capabilities could be augmented by predictive analytics. By integrating machine learning-based clustering, the system could automatically categorize vessel behaviors, while outlier detection algorithms could flag anomalies in port-exit records without requiring manual filtering by the user.

Finally, this project offered a significant learning opportunity. While the initial complexity of the dataset presented a steep learning curve, the process of untangling these records was highly rewarding. It reinforced a the fundamental principle that effective visual analytics is not just about creating appealing graphics, but about crafting tools that empower users to derive insights from complex data. And I must say, it was quite fun to play detective in the process of uncovering illicit fishing operations in Oceanus!